\chapter{Interrogazione del Data Mart}
In questa sezione vengono definite le interrogazioni sui data mart che sono state effettuate.

Tutte le richieste fatte ai data mart vengono, di conseguenza, effettuate attraverso interrogazioni ad un DBMS MySQL.

\subsubsection{Note}
Il modello concettuale e logico indicano la presenza della dimensione "tipo di veicolo", ma al momento della stesura
di questo documento Helbiz ritorna informazioni solo per i veicoli di tipo monopattino.
A questo proposito solo una delle interrogazioni presenti in questo documento utilizzano tale dimensione, a titolo d'esempio.




\section{Visualizzazione utilizzi/distanze settimanali}
Lo scopo di questa interrogazione è di poter avere una panoramica ad alto livello
sugli utilizzi per ogni giorno della settimana; questo dato è utile ad 
Helbiz in maniera da poter predisporre più o meno veicoli ai cittadini.

Questo tipo di predizione permette di identificare ad altissimo livello
in quali giorni possano essere fatte più manutenzioni (togliendo quindi i veicoli dalla strada)
e in quali non ci si può permettere questo lusso.
\begin{figure}[H]                                                                                                                                                            
\centering                                                                                                                                                                   
\includegraphics[width=\textwidth]{images/query1}                                                                                                                                   
\label{fig:query1}                                                                                                                                                           
\end{figure}

\iffalse
SELECT SUM(v.uses)                AS total_uses, 
	   SUM(v.travelled_distance)  AS total_distance, 
       AVG(v.uses)                AS average_uses,
       AVG(v.travelled_distance)  AS average_distance,
       d.weekday
FROM vehicle_use AS v JOIN date AS d ON v.start_time = d.id
GROUP BY d.weekday
ORDER BY FIELD(weekday ,'monday', 'tuesday', 'wednesday',
			   'thursday', 'friday', 'saturday', 'sunday')
\fi

\begin{table}[H]
\centering
\resizebox{\textwidth}{!}{%
\begin{tabular}{|
>{\columncolor[HTML]{FFFFFF}}c |c|c|c|c|}
\hline
\cellcolor[HTML]{3166FF}{\color[HTML]{FFFFFF} \textbf{Giorno}} & \cellcolor[HTML]{3166FF}{\color[HTML]{FFFFFF} \textbf{Totale Utilizzi}} & \cellcolor[HTML]{3166FF}{\color[HTML]{FFFFFF} \textbf{Totale Distanza (Km)}} & \cellcolor[HTML]{3166FF}{\color[HTML]{FFFFFF} \textbf{Media Utilizzi}} & \cellcolor[HTML]{3166FF}{\color[HTML]{FFFFFF} \textbf{Media Distanza (Km)}} \\ \hline
{\color[HTML]{000000} Lunedì}                                  & \cellcolor[HTML]{FF9494}1410                                            & \cellcolor[HTML]{FF8686}907                                                  & \cellcolor[HTML]{FF5555}15                                             & \cellcolor[HTML]{FF6666}9                                                   \\ \hline
{\color[HTML]{000000} Martedì}                                 & \cellcolor[HTML]{FF4040}1714                                            & \cellcolor[HTML]{FF5151}1013                 & \cellcolor[HTML]{FF7F7F}14                                             & \cellcolor[HTML]{FF3333}10                                                  \\ \hline
{\color[HTML]{000000} Mercoledì}                               & \cellcolor[HTML]{FFC8C8}1229                                            & \cellcolor[HTML]{FFB2B2}818                                                  & \cellcolor[HTML]{FFAAAA}13                                             & \cellcolor[HTML]{FF6666}9                                                   \\ \hline
{\color[HTML]{000000} Giovedì}                                 & \cellcolor[HTML]{FFBABA}1279                                            & \cellcolor[HTML]{FFC9C9}771                                                  & \cellcolor[HTML]{FFEFEF}11                                             & \cellcolor[HTML]{FFEFEF}6                                                   \\ \hline
{\color[HTML]{000000} Venerdì}                                 & \cellcolor[HTML]{FF0000}{\color[HTML]{FFFFFF} \textbf{1935}}            & \cellcolor[HTML]{FF0000}{\color[HTML]{FFFFFF} \textbf{1177}}                                                 & \cellcolor[HTML]{FF2A2A}16                                             & \cellcolor[HTML]{FF9999}8                                                   \\ \hline
{\color[HTML]{000000} Sabato}                                  & \cellcolor[HTML]{FFEFEF}1036                                            & \cellcolor[HTML]{FFEFEF}664                                                  & \cellcolor[HTML]{FF7F7F}14                                             & \cellcolor[HTML]{FF6666}9                                                   \\ \hline
{\color[HTML]{000000} Domenica}                                & \cellcolor[HTML]{FF5B5B}1612                                            & \cellcolor[HTML]{FF3838}1060                                                 & \cellcolor[HTML]{FF0000}{\color[HTML]{FFFFFF} \textbf{17}}             & \cellcolor[HTML]{FF0000}{\color[HTML]{FFFFFF} \textbf{11}}                  \\ \hline
\end{tabular}%
}
\end{table}

Si può notare come la domenica ci siano utilizzi più frequenti ma con bassa percorrenza: questo dato potrebbe essere collegato al turismo che di solito è più frequente durante il week end.

Nelle giornate lavorative si ha un graduale discesa dell'utilizzo per poi avere un picco di venerdì: è normale avere spostamenti di questo genere in quanto il venerdì dichiara la fine
della settimana lavorativa e vi è più voglia da parte della gente di frequentare i locali della città .

\begin{figure}[H]                                                                                                                                                            
\centering                                                                                                                                                                   
\includegraphics[width=\textwidth]{images/result12}                                                                                                                                   
\label{fig:result12}                                                                                                                                                           
\end{figure}


\section{Utilizzo per livelli di pioggia mese per mese}
La pioggia è uno dei maggiori deterrenti per impedire alle persone di
utilizzare i veicoli come monopattini o biciclette.

È quindi molto interessante capire a seconda della stagione quale
sia il livello di "sopportazione della pioggia" dell'utenza; come prima,
l'obiettivo è quello di capire in quali circostanze è possibile ritirare
più veicoli dalle strade o se addirittura decidere di sospendere il servizio: in
certe circostanze può essere un costo senza guadagno mantenere attivo l'apparato
di Helbiz se l'utenza è nulla per lunghi periodi di tempo.
Questo significa che grazie a questi dati, se in un determinato periodo futuro
dell'anno si dovessero presentare N giorni di pioggia con un livello maggiore ad M,
allora è conveniente sospendere il servizio.
\begin{figure}[H]                                                                                                                                                            
\centering                                                                                                                                                                   
\includegraphics[width=\textwidth]{images/query2}                                                                                                                                   
\label{fig:query2}                                                                                                                                                           
\end{figure}
\iffalse
SELECT SUM(v.uses)                AS total_uses, 
	   SUM(v.travelled_distance)  AS total_distance,
       w.rain_level,
       d.month
FROM vehicle_use AS v JOIN weather as w ON v.weather = w.id
					  JOIN date AS d ON v.start_time = d.id
GROUP BY w.rain_level, d.month
ORDER BY w.rain_level DESC
\fi





\section{Confronto degli utilizzi durante uno sciopero per fascia oraria per tipo di veicolo}
Forse uno dei dati più importanti non solo Helbiz ma anche per il comune e 
i cittadini stessi; grazie a queste informazioni Helbiz può gestire durante 
un periodo critico come quello di sciopero il numero di mezzi da mettere a disposizione,
pianificando per ogni ora la strategia migliore per l'immissione di ulteriori veicoli.
\begin{figure}[H]                                                                                                                                                            
\centering                                                                                                                                                                   
\includegraphics[width=\textwidth]{images/query3}                                                                                                                                   
\label{fig:query3}                                                                                                                                                           
\end{figure}
\iffalse
SELECT SUM(v.uses) AS total_use, d.hour, v.type
FROM vehicle_use AS v JOIN strike AS s ON s.id = v.strike
				 JOIN dates AS d ON v.start_date = d.id
GROUP BY d.hour, v.type
\fi

Purtroppo ad oggi per la città di Torino non sono presenti dati relativi alle
biciclette ma esclusivamente ai monopattini.

\begin{figure}[H]                                                                                                                                                            
\centering                                                                                                                                                                   
\includegraphics[width=\textwidth]{images/result3}                                                                                                                                   
\label{fig:result3}
\caption{Andamento degli utilizzi lungo la giornata durante uno sciopero}                                                                                                                                                                                                                                                                                                                     
\end{figure}


\section{Delta utilizzi al variare delle temperature a dipendere della presenza di scioperi}
Un altro deterrente per gli utenti è la temperatura; ma la presenza di scioperi potrebbe
far crescere il numero della domanda in quanto un utente è più predisposto a utilizzare un 
monopattino durante una giornata fredda se si trova senza possibilità di scelta.

\begin{figure}[H]                                                                                                                                                            
\centering                                                                                                                                                                   
\includegraphics[width=\textwidth]{images/query4}                                                                                                                                   
\label{fig:query4}                                                                                                                                                           
\end{figure}
\iffalse
SELECT t1.avg_without_strike, t2.avg_with_strike, t1.temperature_level
FROM (
		SELECT AVG(v.uses) AS avg_without_strike, 
  			   w.temperature_level
  		FROM vehicle_use AS v JOIN weather AS w ON v.weather = w.id
  		WHERE v.strike IS NULL
  		GROUP BY w.temperature_level
      ) AS t1 
      	
        CROSS JOIN
      
      (
        SELECT AVG(v.uses) AS avg_with_strike, 
               w.temperature_level
  		FROM vehicle_use AS v JOIN weather AS w ON v.weather = w.id
        WHERE v.strike IS NOT NULL
  		GROUP BY w.temperature_level
      ) AS t2
        ON t1.temperature_level = t2.temperature_level
ORDER BY t1.temperature_level ASC
\fi

Come prevedibile, in entrambi i casi gli utilizzi calano non appena si presenta una minima precipitazione.
Al livello massimo di precipitazioni abbiamo 0 utilizzi in entrambi i casi, poiché tale evento non si è mai verificato.
Al livello precedente compare un utilizzo durante uno sciopero, ma si tratta di un unico caso in tutto il dataset; per questo motivo può essere
un dato trascurabile fino a che non ci saranno ulteriori evidenze.

\begin{figure}[H]                                                                                                                                                            
\centering                                                                                                                                                                   
\includegraphics[width=\textwidth]{images/result4}                                                                                                                                   
\label{fig:result4}
\caption{Utilizzi in base alle precipitazioni: in blu durante un sciopero, in arancione in una giornata senza sciopero}                                                                                                                                                           
\end{figure}



\section{Numero di utilizzi durante uno sciopero nelle ore di punta}
Questa interrogazione analizza ancora più nel dettaglio la situazione durante
uno sciopero e serve a capire quale sia l'effettiva domanda
da parte dell'utenza nelle fasce orarie più critiche durante uno sciopero.
Queste fasce sono quelle identificate come "orario di punta" e sono 
determinate dall'alto traffico generato dai lavoratori durante il tragitto casa-lavoro e lavoro-casa.
\begin{figure}[H]                                                                                                                                                            
\centering                                                                                                                                                                   
\includegraphics[width=\textwidth]{images/query5}                                                                                                                                   
\label{fig:query5}                                                                                                                                                           
\end{figure}
\iffalse
SELECT SUM(v.uses) AS total_uses, 
       d.hour,
       CASE 
       		WHEN (d.hour >= 7 AND d.hour <= 9)   
           	  OR (d.hour >= 18 AND d.hour <= 19) THEN "true"
       		ELSE 								      "false"
       END AS rush_hour
FROM vehicle_use AS v JOIN date AS d ON v.start_time = d.id
WHERE strike IS NOT NULL
GROUP BY d.hour
\fi

Si nota subito che il 41\% degli utilizzi del giorno viene fatto solo in queste 5 ore.
\begin{figure}[H]                                                                                                                                                            
\centering                                                                                                                                                                   
\includegraphics[width=\textwidth]{images/result5}                                                                                                                                   
\label{fig:result5} 
\caption{Durante le ore di punta vengono fatti più utilizzi che non durante i normali orari, con un rapporto di circa 8:3 }                                                                                                                                                          
\end{figure}
Risulta quindi vitale per Helbiz riuscire a gestire i due periodi che vanno dalle 7 alle 9  e dalle 18 alle 19.


\section{Numero di utilizzi durante diverse fasce orarie}
Questa interrogazione, molto simile alla precedente, invece analizza
tutte le fasce orarie indipendentemente dal fatto che ci sia o meno uno sciopero.
Le fasce orarie identificate sono
\begin{itemize}
	\item{\textbf{Orario di punta:} come spiegato nella sezione precedente, è l'orario di maggior traffico causato dai lavoratori}
	\item{\textbf{Orario lavorativo}}
	\item{\textbf{Uscita da scuola} è l'orario di uscita per la maggior parte degli studenti della scuola dell'obbligo}
	\item{\textbf{Uscita da scuola pomeridiana} simile a quella precedente, ma riguarda la fascia pomeridiana}
\end{itemize}

Queste fasce non sono solo utili a capire il numero di veicoli da mettere a disposizione, ma indicano grossolanamente quale sia 
la fascia d'età degli utilizzatori. Questa informazione diventa molto utile in fase promozionale del servizio (\emph{e.g.} se nella fascia d'orario
uscita scuole non si nota abbastanza affluenza, si potrebbe pensare di creare una agevolazione in collaborazione con le scuole).

\begin{figure}[H]                                                                                                                                                            
\centering                                                                                                                                                                   
\includegraphics[width=\textwidth]{images/query6}                                                                                                                                   
\label{fig:query6}                                                                                                                                                           
\end{figure}
\iffalse
SELECT SUM(v.uses) AS total_uses,
	   AVG(v.uses) AS average_uses,
       d.hour,
       d.minute,
       CASE 
       		WHEN (d.hour >= 7 AND d.hour <= 9)   
           	  OR (d.hour >= 18 AND d.hour <= 19) THEN "true"
       		ELSE 								      "false"
       END AS rush_hour,
       CASE 
       		WHEN (d.hour >= 9 AND d.hour <= 18)  THEN "true"
       		ELSE 								      "false"
       END AS working_hour,
       CASE 
       		WHEN (d.hour >= 12 AND d.hour <= 14)  THEN "true"
       		ELSE 								      "false"
       END AS school_exit_time,
       CASE 
       		WHEN (d.hour >= 16 AND d.hour <= 17)  THEN "true"
       		ELSE 								      "false"
       END AS school_exit_time2
FROM vehicle_use AS v JOIN date AS d ON v.start_time = d.id
WHERE strike IS NULL
  AND d.weekday NOT IN ('saturday', 'sunday')
GROUP BY d.hour, d.minute
\fi

È più facile in questa maniera visualizzare la richiesta all'interno di ogni singola fascia oraria.
Notasi che questa query potrebbe dare maggior valore ad Helbiz applicando un ulteriore filtro sul periodo delle due o tre settimane precedenti,
in maniera tale che gli accumuli non rendano ininfluenti gli sviluppi più recenti.

\begin{figure}[H]                                                                                                                                                            
\centering                                                                                                                                                                   
\includegraphics[width=\textwidth]{images/result6}                                                                                                                                   
\label{fig:result6}
\caption{Utilizzi divisi per fascia oraria: orario di punta (7-9 e 18-19), orario lavorativo (9-18), uscita da scuola (12-14) e uscita da scuola pomeridiana (16-17)}                                                                                                                                                           
\end{figure}
