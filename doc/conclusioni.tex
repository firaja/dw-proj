\chapter{Conclusioni e sviluppi futuri}
Le analisi condotte in questo caso di studio permettono di valutare l'andamento degli 
utilizzi dei monopattini con differenti punti di vista anche grazie all’integrazione della sorgente dati
Meteo. Helbiz in quanto azienda sicuramente vorrà espandersi il più possibile sul suolo nazionale italiano;
in questo senso ha reso disponibile dagli inizi di Febbraio 2020 le API anche per Milano e Roma.
Questo implica l'integrazione con una base di dati che fornisca la lista di comuni sempre aggiornata, divisi per 
provincia e regioni: il cambiamento dell'appartenenza di un comune rispetto ad una provincia, la soppressione o aggiunta
di una provincia o addirittura la annessione di un comune ad una regione è un evento non così raro quanto sembri; per questo
c'è bisogno di un aggiornamento costante, con un refresh rate ovviamente molto basso. \\~\\


La collaborazione con Torinometeo potrebbe quindi espandersi verso l'integrazione con più sorgenti meteorologiche locali
oppure l'affidarsi ad un unico servizio come OpenWeatherMap, che però ad oggi risulta essere a pagamento dopo il superamento di una soglia di chiamate per minuto.
Sarà quindi da valutare il costo (in termini di tempo, denaro e risorse) di una unica integrazione a pagamento o di
molteplici ma diverse integrazioni.\\~\\


Come già descritto nel documento, ad oggi Helbiz rilascia informazioni solo relative al ai monopattini.
Una volta che anche posizioni sulle biciclette saranno disponibili sarà possibile tracciare dei pattern comportamentali
divisi per tipo di veicolo.\\~\\


L'utilizzo risulta essere l'unico fatto identificato nel data warehouse creato. Se Helbiz mettesse a disposizione i dati anagrafici
(opportunamente "anonimizzati" in conformità alla normativa vigente europea sulla privacy) si potrebbero aggiungere
maggiori dettagli sull'utenza e determinare quali fasce d'età siano più attratte dal servizio.

