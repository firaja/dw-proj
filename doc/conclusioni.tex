\chapter{Conclusioni e sviluppi futuri}

Le analisi condotte in questo caso di studio permettono di valutare l'andamento degli 
utilizzi dei monopattini da diversi punti di osservazione anche grazie all’integrazione delle sorgenti dati
di meteo e scioperi.

Helbiz in quanto azienda in espansione a guida italiana mira a coprire la penisola con almeno una presenza per regione; 
fa da apripista verso il raggiungimento di tale obiettivo la rinnovata presenza del servizio a Milano dopo circa 6 mesi di sosta dettata dal gap legislativo.
A tal fine, proprio sull'esempio di Milano, potrebbe essere utile integrare oltre alle sorgenti di cui si dispone degli open data sull'utilizzo di ZTL,
Area C e Area B al fine di vedere se tali mezzi costituiscono un'alternativa ai mezzi pubblici per l'accesso a tali aree contingentate. \\

La collaborazione con Torinometeo potrebbe espandersi verso l'integrazione di più sorgenti meteorologiche locali;
in alternativa si potrebbe sceglie un unico servizio come OpenWeatherMap, il quale però ad oggi risulta essere a pagamento dopo il superamento di una soglia di
massima di richieste per minuto e non permette l'accesso a dati storici a meno di sottoscrivere un abbonamento.
Rimane quindi da valutare il costo (in termini di tempo, denaro e risorse) dell'integrazione e successivo utilizzo di un'unica sorgente dati a pagamento o di
molteplici sorgenti dati gratuite quali Torinometeo al costo di dover effettuare diverse integrazioni, con relativa complicazione della fase di
riconciliazione degli schemi delle diverse sorgenti.\\

Come già anticipato nel corso della relazione il progetto è stato eseguito profilando le posizioni dei monopattini ma progettando un data warehouse potenzialmente
in grado di gestire qualsiasi altro tipo di veicolo adibito al noleggio.
Ciò è dovuto al fatto che, ad oggi, Helbiz permette il noleggio di monopattini in tutte le città in cui è presente ma restringe il noleggio di altri veicoli come
le biciclette ad una cerchia ristretta di città pilota.
Una volta che il noleggio delle biciclette diverrà capillare alle città già servite dai monopattini, disponendo delle posizioni sulle biciclette,
sarà possibile tracciare dei pattern comportamentali divisi per tipo di veicolo.\\

L'utilizzo risulta essere l'unico fatto identificato nel data warehouse creato. Se Helbiz mettesse a disposizione oltre che le statistiche sull'uso dei veicoli
anche i dati anagrafici degli utilizzatori degli stessi (opportunamente "anonimizzati" in conformità alla normativa europea vigente in termini di privacy),
si potrebbero aggiungere maggiori dettagli sull'utenza e determinare quali fasce d'età fanno un uso più intenso del servizio.
