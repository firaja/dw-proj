\chapter{Definizione del Datawarehouse}
\section{Carico di lavoro}
\begin{table}[h]
\centering
\begin{tabular}{|l|l|}
\hline
\rowcolor[HTML]{3166FF} 
{\color[HTML]{FFFFFF} \textbf{Fatto}} & {\color[HTML]{FFFFFF} \textbf{Interrogazione}}                                  \\ \hline
                                      & Numero di utilizzi durante uno sciopero nelle ore di punta                      \\ \cline{2-2} 
                                      & Numero di utilizzi durante una giornata di pioggia                              \\ \cline{2-2} 
                                      & Numero di utilizzi durante le fasce orarie lavorative                           \\ \cline{2-2} 
                                      & Numero di utilizzi durante le fasce orarie lavorative                           \\ \cline{2-2} 
                                      & Delta del numero di utilizzi tra una giornata di sciopero e una non di sciopero \\ \cline{2-2} 
                                      & Delta del numero di utilizzi tra una giornata di pioggia e una di sole          \\ \cline{2-2} 
\multirow{-7}{*}{Corsa}               & Delta del numero di utilizzi tra una gioranta di pioggia e una di sciopero      \\ \hline
\end{tabular}
\end{table}
\section{Progettazione concettuale}
La progettazione concettuale comporta l’utilizzo dei requisiti identificati nella Sezione xyz
per realizzare uno schema concettuale per il data mart. A tale scopo viene utilizzato il
Dimensional Fact Model (DFM), un modello concettuale creato appositamente per supportare la progettazione di data mart. 
In questa Sezione viene descritta la progettazione del DFM che modella il fatto relativo ad un ascolto.

\section{Progettazione Logica e Fisica}
In questa Sezione viene descritto il processo di trasformazione del DFM, descritto al paragrafo precedente, in modello logico. 
Innanzitutto va specificato che è stata applicata la tecnica ROLAP (Relational On-Line Analytical Processing) che prevede l’utilizzo di un
modello relazionale per la rappresentazione dei dati multidimensionali.