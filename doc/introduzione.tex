\chapter{Introduzione}

La ricerca di stili di vita sempre più sostenibili e a basso impatto per
l'ambiente ha influenzato il comportamento di ogni singolo individuo e con esso
la mobilità urbana.
Ciò ha portato alla condivisione dei mezzi di trasporto privati e alla rinnovata
diffusione di mezzi quali la bicicletta e il monopattino insieme alle loro versioni
elettriche.
La capillare distribuzione degli smartphone e delle connessioni dati a basso costo
hanno permesso ad aziende private di creare servizi di sharing che consentono di 
noleggiare a costi contenuti un mezzo per un singolo tragitto in modalità free flow, ovvero
senza l'onere di riportare il mezzo dove lo si è preso o presso appositi luoghi
di raccolta ma parcheggiando lo stesso, entro un'area prestabilita, presso la
propria destinazione.

Una tra le aziende in questione è \emph{Helbiz}, che permette il noleggio di monopattini
elettrici e/o biciclette, a seguito dell'iscrizione al relativo servizio online,
utilizzando l'apposita applicazione direttamente dal proprio smartphone. Nonostante un'iniziale
successo del servizio abbia portato ad una veloce diffusione nelle principali provincie
italiane, l'assenza di regolamentazione nel codice della strada per tali mezzi di
trasporto ha portato diversi comuni italiani a bandirli o a confinarli a zone adibite
ai soli pedoni, costrigendo la società in questione a rimuovere temporaneamente i
propri monopattini in attesa di una specifica regolamentazione nazionale.
Va sottolineato come a seguito dell'ampia adozione di monopattini e hoverboard
elettrici da parte di privati, il Governo Italiano abbia attenzionato il problema
ed un Decreto Ministeriale sia stato pubblicato all'inizio di Gennaio 2020 allo
scopo di paragonare tali veicoli ai velocipedi.
Relativamente alla sospensione del servizio, la situazione non
è però comune a tutti i comuni italiani, infatti città quali Roma, Torino e Verona
hanno emesso propri regolamenti che hanno permesso ad Helbiz di continuare a
noleggiare i propri veicoli.

Un servizio di sharing mobility quale quello sopra esposto si prefigge di risolvere
alcuni problemi per i propri utenti quali la copertura assente o scarsa dei mezzi
pubblici in un determinato tratto o i problemi di traffico delle ore di punta, il tutto
consentendo l'accesso a piste ciclabili, ZTL o zone con accesso a pagamento per veicoli
a motore, ma gratuito per chi utilizza il servizio in oggetto.

Altro fattore di primaria importanza per la mobilità sono le condizioni metereologiche,
in quanto esse influenzano gli spostamenti oltre che la sicurezza del mezzo di
trasporto utilizzato, guidando spesso la scelta del mezzo di trasporto per gli utenti
finali. 

Una problematica che coinvolge i pendolari, siano essi per motivi lavorativi o di
studio, è la sempre più frequente presenza di scioperi dei mezzi pubblici, in numero
di almeno uno al mese e il più delle volte con ristrette fasce di garanzia.

Lo scopo del progetto illustrato in questo documento è quello di creare un data
warehouse che contenga i dati sull'utilizzo dei monopattini di Helbiz, i dati sulle
condizioni metereologiche e quelli sugli scioperi, al fine di analizzare come tali
eventi influiscono sugli spostamenti di chi vive lavora o studia nella città di Torino.
